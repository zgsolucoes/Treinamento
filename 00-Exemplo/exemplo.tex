\documentclass{beamer}

\usetheme{zg}

\graphicspath{{./fig/}}

\title{Exemplo de Apresentação}
\date{\today}
\author{Fernando Camargo}
\institute{ZG Soluções}


\begin{document}
  \maketitle
  
  \section{Slides básicos}
  
  \begin{frame}{O Problema do Planejamento Energético de Sistemas\\Hidrotérmicos}
    
    Considerações:
    \begin{outline}
      \1<1-> Decisões de operação afetam decisões futuras: reservatórios
      \1<2-> Acoplamento de usinas na mesma bacia
      \1<3-> Aleatoriedade das vazões
      \1<4-> Objetivos:
	\2<4-> Minimizar custo de produção (custo térmico)
	\2<5-> Aumentar produção hidrelétrica para diminuir termelétrica
	\2<6-> Definir uma estratégia de geração para cada usina de interconectada
	\2<7-> Atender demanda com confiabilidade
    \end{outline}
  \end{frame}
  
  \begin{frame}{O Problema do Planejamento Energético de Sistemas\\Hidrotérmicos}
    
    Características:
    \begin{outline}
      \1<1-> Problema dinâmico de otimização envolvendo o tempo
      \1<2-> Problema não separável
      \1<3-> Função de objetivo não linear e não convexa
      \1<4-> Problema estocástico
      \1<5-> De grande porte
    \end{outline}
    \onslide<6->{Adoção de único modelo inviável.\\}
    \onslide<7->{\alert{Solução: decomposição do problema.}}
  \end{frame}
  
  \begin{frame}{Algoritmos utilizados}
    
    \begin{outline}
    \0 \onslide<1->{Algoritmos:}
      \1<1-> Programa Dinâmica e Estocástica
	\2 Uma tabela de soluções ótimas para cada estado do sistema é gerada
	\2 Explosão combinatorial de estados
	\2 Tentativas de simplificação do problema uso dessa técnica
      \1<2-> Técnicas não lineares baseadas na teoria lagrangeana
      \1<2-> Técnicas não lineares por fluxos de redes
    \0 \onslide<3->{Problemas:}
      \1<3-> Não garantia de ótimo global
      \1<4-> Problemas de convergência 
      \1<5-> Computabilidade
    \end{outline}
  \end{frame}
  
  \section{Tabelas e Figuras}
  
  \begin{frame}{Matriz energética brasileira}
    A Tabela \ref{table:dependenciaNatureza2012} de \cite{junior1998} mostra como a atual matriz brasileira de produção de energia elétrica é muito dependente dos fluxos da Natureza.
    
    \begin{table}
    \caption{Dependência da Natureza para geração de energia elétrica (Fonte: ANEEL (2012))}
    \label{table:dependenciaNatureza2012}
    \begin{tabular}{|l|l|l|l|}
	\hline
	Hidro       & 84.094,7   & Térmica      & 32.730,8  \\ \hline
	Eólica      & 1.820,3    & Nuclear      & 2.007,0    \\ \hline
	Total       & 85.915     & Total        & 34.737,8  \\ \hline
	\% do total & 71,2\%     & \% do total  & 28,8\%    \\ \hline
    \end{tabular}
    \end{table}
  \end{frame}
  
  \begin{frame}{Principais componentes de uma Usina Hidrelétrica}
    
    \begin{figure}
      \centering
      \includegraphics[width=\textwidth]{./fig/Componentes-usinas-hidreletricas.jpg}
      \label{fig:componentesUsinasHidreletricas}
    \end{figure}
  \end{frame}
  
  \section{Ambientes Matemáticos}
  
  \begin{frame}{O que são números primos?}
    \begin{definition}
      Um \alert{número primo} é um número que possui exatamente dois divisores.
    \end{definition}
    \begin{example}
      \begin{itemize}
      \item 2 é primo (dois divisores: 1 e 2).
      \item 3 é primo (dois dividores: 1 e 3).
      \item 4 não é primo (\alert{três} divisores: 1, 2, e 4).
      \end{itemize}
    \end{example}
  \end{frame}
  
  \begin{frame}{Não existe o maior número primo}
    \begin{theorem}
      Não existe o maior número primo.
    \end{theorem}
    \begin{proof}
      \begin{enumerate}
	\item Suponha que $p$ fosse o maior número primo.
	\item Seja $q$ o produto dos primeiros $p$ números.
	\item Então $q + 1$ não é divisível por nenhum deles.
	\item Mas $q + 1$ é maior que $1$, portanto divisível por algum número primo não existente nos primeiros $p$ números.\qedhere
      \end{enumerate}
    \end{proof}
    \uncover<2->{A prova usa \textit{reductio ad absurdum}.}
  \end{frame}
  
  \section{Referências}
  
  \begin{frame}[allowframebreaks]{Referências}
    %\nocite{*} % Se quiser que todas citações apareçam
    \bibliography{./bib/exemplo.bib}
  \end{frame}

\end{document}
