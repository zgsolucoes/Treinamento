\documentclass{beamer}

\usetheme{zg}

\title{Gradle}
\date{\today}
\author{Fernando Camargo}
\institute{ZG Soluções}


\begin{document}
\maketitle

\section{Por que uma ferramenta de Build?}

\begin{frame}{Por que uma ferramenta de Build?}
 \begin{outline}
   \1<1-> Projeto independente de IDE
   \1<2-> Automatização de build
 \end{outline}
\end{frame}

\section{Ant, Maven e Gradle}

\subsection{Ant}

\begin{frame}{Apache Ant}
 \begin{outline}
   \1<1-> Primeira build tool para Java
   \1<2-> Extrema flexibilidade
   \1<3-> Não impõe convenções em projetos Java
 \end{outline}
\end{frame}

\begin{frame}[fragile]{Exemplo de Ant}
 \begin{minted}[fontsize=\tiny]{xml}
<project>

    <target name="clean">
        <delete dir="build"/>
    </target>

    <target name="compile">
        <mkdir dir="build/classes"/>
        <javac srcdir="src" destdir="build/classes"/>
    </target>

    <target name="jar">
        <mkdir dir="build/jar"/>
        <jar destfile="build/jar/HelloWorld.jar" basedir="build/classes">
            <manifest>
                <attribute name="Main-Class" value="oata.HelloWorld"/>
            </manifest>
        </jar>
    </target>

    <target name="run">
        <java jar="build/jar/HelloWorld.jar" fork="true"/>
    </target>

</project>
  \end{minted}
\end{frame}

\begin{frame}{Problemas do Ant}
 \begin{outline}
   \1<1-> Flexível demais $\rightarrow$ projetos não possuem estrutura padrão
   \1<2-> Muito verboso $\rightarrow$ escreve-se muito para uma build simples
   \1<3-> Não possui gerenciamento de dependências
 \end{outline}
\end{frame}

\subsection{Maven}

\begin{frame}{Apache Maven}
 \begin{outline}
   \1<1-> Convenção sobre Configuração $\rightarrow$ escreve-se pouco para uma build simples
   \1<2-> Gerenciamento de dependências com resolução de dependências transitivas
 \end{outline}
\end{frame}

\begin{frame}{Estrutura de diretórios}
 \begin{table}[]
    \begin{tabular}{@{}ll@{}}
      \toprule
      Diretório           & Função                            \\ \midrule
      src/main/java       & Código fonte                      \\
      src/main/resources  & Recursos não compilados           \\
      src/test/java       & Código de testes                  \\
      src/test/resources  & Recursos não compilados de testes \\
      src/main/webapp     & Recursos WEB                      \\
      target              & Resultados de build               \\ \bottomrule
    \end{tabular}
  \end{table}
\end{frame}




\section{Gradle}

\subsection{DSL}

\subsection{Projetos e Tarefas}

\subsection{Plugins}

\subsection{Gerenciamento de dependências}

\subsection{Multiplos Projetos}

\section{Conclusões}


\end{document}