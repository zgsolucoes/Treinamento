\documentclass{beamer}

\usetheme{zg}

\title{Spock Framework}
\date{\today}
\author{Fernando Camargo}
\institute{ZG Soluções}


\begin{document}
\maketitle

\begin{frame}{Apresentação}
 \begin{outline}
   \1 Fernando Camargo
   \1 Mestre em Engenharia de Computação na UFG
   \1 Dev na ZG Soluções
   \1 Artigos publicados na Java Magazine
 \end{outline}
\end{frame}

\begin{frame}
  \begin{center}
    \includegraphics[width=0.8\textwidth]{hackathon}
  \end{center}
\end{frame}

\section{Spock?}

\begin{frame}
  \begin{center}
    \includegraphics[width=\textwidth]{spock.jpg}
  \end{center}
\end{frame}

\begin{frame}{Spock Framework}
 \begin{outline}
   \1<1-> Framework de testes
   \1<2-> Para aplicações da Plataforma Java
   \1<3-> Testes feitos em Groovy
   \1<4-> Compatível com JUnit
 \end{outline}
\end{frame}

\begin{frame}{Objetivos}
 \begin{outline}
   \1<1-> Menor código de testes
   \1<2-> Legibilidade de testes
   \1<3-> Testes $\rightarrow$ Especificações $\rightarrow$ Documentação
   \1<4-> Extensibilidade
 \end{outline}
\end{frame}

\section{State Based Testing}

\begin{frame}{State Based Testing}
 \begin{outline}
   \1<1-> Testes que invocam métodos e validam o estado do objeto sob testes
   \1<2-> BDD (Behavior-Driven Development):
    \2<2-> Arranjar (Given)
    \2<2-> Agir (When)
    \2<2-> Verificar (Then)
 \end{outline}
\end{frame}

\begin{frame}[fragile]{Teste com JUnit}
 \begin{minted}[fontsize=\scriptsize]{java}
public class AccountTest {
  @Test
  public void withdrawSomeAmount(){
    // given
    Account account = new Account(BigDecimal.valueOf(5));
    
    // when
    account.withdraw(BigDecimal.valueOf(2));
    
    // then
    assertEquals(BigDecimal.valueOf(3), account.getBalance());
  }
  
}
  \end{minted}
\end{frame}

\begin{frame}[fragile]{Teste com Spock}
 \begin{minted}[fontsize=\scriptsize]{groovy}
class AccountSpec extends Specification {
  
  def "withdrawing some amount decreases the balance by that amount"(){
    given:
    Account account = new Account(BigDecimal.valueOf(5))
    
    when:
    account.withdraw(BigDecimal.valueOf(2))
    
    then:
    account.getBalance() == BigDecimal.valueOf(3) 
  }
  
}
  \end{minted}
\end{frame}

\begin{frame}[fragile]{Teste com Spock}
 \begin{minted}[fontsize=\scriptsize]{groovy}
class AccountSpec extends Specification {
  
  def "withdrawing some amount decreases the balance by that amount"(){
    given:
    Account account = new Account(5.0)
    
    when:
    account.withdraw(2.0)
    
    then:
    account.balance == 3.0
  }
  
}
  \end{minted}
\end{frame}

\begin{frame}[fragile]{Teste com Spock}
 \begin{minted}[fontsize=\scriptsize]{groovy}
class AccountSpec extends Specification {
  
  def "withdrawing some amount decreases the balance by that amount"(){
    given: "an account with a balance of five euros"
    Account account = new Account(5.0)
    
    when: "two euros are withdrawn"
    account.withdraw(2.0)
    
    then: "three euros remain in the account"
    account.balance == 3.0
  }
  
}
  \end{minted}
\end{frame}

\begin{frame}[fragile]{Teste de Exception com JUnit}
 \begin{minted}[fontsize=\scriptsize]{java}
public class AccountTest {

  @Test
  public void withdrawNegativeAmount(){
    // given
    Account account = new Account(BigDecimal.valueOf(5));
    
    try {
      // when
      account.withdraw(BigDecimal.valueOf(-1));
      
      // then?
      fail("Should have thrown NegativeAmountWithdrawnException")
    }
    catch(NegativeAmountWithdrawnException e){
      // then?
      assertEquals(BigDecimal.valueOf(-1), e.getAmount());
    }
  }
  
}
  \end{minted}
\end{frame}

\begin{frame}[fragile]{Teste de Exception com Spock}
 \begin{minted}[fontsize=\scriptsize]{groovy}
class AccountSpec extends Specification {
  
  def "can't withdraw a negative amount"(){
    given: "an account with a balance of five euros"
    Account account = new Account(5.0)
    
    when: "trying to withdraw -1"
    account.withdraw(-1.0)
    
    then: "an exception is thrown"
    NegativeAmountWithdrawnException e = thrown()
    e.amount == -1.0
  }
  
}
  \end{minted}
\end{frame}

\section{Conceitos básicos}

\begin{frame}[fragile]{Specification}
 \begin{minted}[fontsize=\scriptsize]{groovy}
import spock.lang.*

class MyFirstSpecification extends Specification {
  // fields
  def col = new Collaborator()
  @Shared
  def sharedCol = new Collaborator()
  
  // fixture methods
  def setup() {}
  def cleanup() {}
  def setupSpec() {}
  def cleanupSpec() {}

  // feature methods
  // helper methods
}
  \end{minted}
\end{frame}

\begin{frame}{Ciclo de vida e blocos}
  \begin{center}
    \includegraphics[width=0.8\textwidth]{blocos}
  \end{center}
\end{frame}

\begin{frame}{Blocos de Setup}
 \begin{outline}
   \1<1-> Primeiro bloco de um método de testes
   \1<2-> Faz as preparações necessárias para o teste
   \1<3-> Pode começar com \alert{setup:}, \alert{given:} ou implícito
 \end{outline}
\end{frame}

\begin{frame}{Blocos de Estímulo e Resposta}
 \begin{outline}
   \1<1-> \alert{when:}
    \2<1-> Descreve um estimulo (invocação de método(s))
    \2<2-> Pode conter código arbitrário
   \1<3-> \alert{then:}
    \2<3-> Descreve uma resposta
    \2<4-> Restrito a condições, condições de exceção, interações e definições de variável
   \1<5-> \alert{expect:}
    \2<5-> Descreve estímulo e resposta numa única expressão
  \0\visible<6->{\alert{when-then} $\rightarrow$ métodos com efeitos colaterais/alteram estado}
  \0\visible<6->{\alert{expect} $\rightarrow$ métodos funcionais}
 \end{outline}
\end{frame}

\begin{frame}[fragile]{Blocos de Estímulo e Resposta}
 \begin{minted}[fontsize=\scriptsize]{groovy}
import spock.lang.*

class StimulusResponseSpecification extends Specification {

  def "when-then style"(){
    when:
    int x = Math.max(5, 9)
    
    then:
    x == 9
  }
  
  def "expect style"(){
    expect:
    Math.max(5, 9) == 9
  }
  
}
  \end{minted}
\end{frame}

\begin{frame}{Blocos de Cleanup}
 \begin{outline}
   \1<1-> Último bloco de um método de testes (podendo ser seguido apenas por where)
   \1<2-> Limpeza de recursos externos
   \1<3-> Sempre começa com \alert{cleanup:}
 \end{outline}
\end{frame}

\section{Data Driven Testing}

\begin{frame}{Data Driven Testing}
 \begin{outline}
   \1<1-> Testa o mesmo comportamento com múltiplas entradas e saídas
   \1<2-> O mesmo teste é executado múltiplas vezes para cada par entrada/saída
   \1<3-> @Unroll faz reportar como testes diferentes
 \end{outline}
\end{frame}

\begin{frame}[fragile]{Teste de Account com Spock (5)}
 \begin{minted}[fontsize=\scriptsize]{groovy}
class AccountSpec extends Specification {
  
  def "withdrawing some amount decreases the balance by that amount"(){
    given:
    Account account = new Account(balance)
    
    when:
    account.withdraw(withdrawn)
    
    then:
    account.balance == remaining
    
    where:
    balance | withdrawn || remaining
    5.0     | 2.0       || 3.0
    4.0     | 0.0       || 4.0
    4.0     | 4.0       || 0.0
  }
  
}
  \end{minted}
\end{frame}

\section{Interaction Based Testing}

\begin{frame}{Interaction Based Testing}
 \begin{outline}
   \1<1-> Comunicação entre objetos (invocação de métodos)
   \1<2-> Spock possui framework próprio de Mocking
 \end{outline}
\end{frame}

\begin{frame}[fragile]{Teste de Publisher/Subscriber com Spock (1)}
 \begin{minted}[fontsize=\scriptsize]{groovy}
class PublisherSubscriberSpec extends Specification {

  def "deliver messages to all subscribers"(){
    given:
    Publisher pub = new Publisher()
    Subscriber sub1 = Mock()
    Subscriber sub2 = Mock()
    pub.subscribers.addAll([sub1, sub2])
    
    when:
    pub.publish("msg")
    
    then:
    1 * sub1.receive("msg")
    1 * sub2.receive("msg")
  }
  
}
  \end{minted}
\end{frame}

\begin{frame}{Mocking/Stubbing com Spock}
 \begin{outline}
   \1<1-> Criação
    \2 \mintinline{groovy}{def sub = Mock(Subscriber)}
    \2 \mintinline{groovy}{Subscriber sub = Mock()}
   \1<2-> Mocking (verificação)
    \2<2-> \mintinline{groovy}{1 * sub.receive("msg")}
    \2<3-> \mintinline{groovy}{(1..3) * sub.receive("msg")}
    \2<4-> \mintinline{groovy}{(1.._) * sub.receive(_ as String)}
    \2<5-> \mintinline{groovy}{1 * sub.receive(!null)}
    \2<6-> \mintinline{groovy}{1 * sub.receive({it.contains("m")})}
    \2<7-> \mintinline{groovy}{1 * _./rec.*/("msg")}
 \end{outline}
\end{frame}

\begin{frame}{Mocking/Stubbing com Spock}
 \begin{outline}
   \1<1-> Stubbing (sem verificação)
    \2<1-> \mintinline{groovy}{sub.receive(_) >> "ok"}
    \2<2-> \mintinline{groovy}{sub.receive(_) >>> ["ok", "ok", "fail"]}
    \2<3-> \mintinline{groovy}{sub.receive(_) >>> {msg -> msg ? "ok" : "fail"}}
   \1<4-> Mocking + Stubbing
    \2<2-> \mintinline{groovy}{3 * sub.receive(_) >>> ["ok", "ok", "fail"]}
 \end{outline}
\end{frame}

\section{Conclusões}

\begin{frame}{Spock vs JUnit}
  \begin{table}[]
    \begin{tabular}{@{}ll@{}}
      \toprule
      Spock               & JUnit              \\ \midrule
      Specification       & Test class         \\
      setup()             & @Before            \\
      cleanup()           & @After             \\
      setupSpec()         & @BeforeClass       \\
      cleanupSpec()       & @AfterClass        \\
      Feature             & Test               \\
      Feature method      & Test method        \\
      Data-driven feature & Theory             \\
      Condition           & Assertion          \\
      Exception condition & @Test(expected=…​)  \\
      Interaction         & Mock expectation   \\ \bottomrule
    \end{tabular}
  \end{table}
\end{frame}

\begin{frame}{Conclusões}
 \begin{outline}
   \1<1-> Código mais conciso
   \1<2-> Funcionalidades do Groovy para os testes
   \1<3-> Documentação através de especificações
 \end{outline}
\end{frame}


\end{document}